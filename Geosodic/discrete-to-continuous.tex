\section{Bridging the Discrete and Continuous: Rigorous Immutable Wave Sampling}
\label{sec:immutable-wave-sampling}

We now extend the meltdown-free framework from discrete enumerations
(\S\ref{sec:enumeration}) to approximating \emph{continuous waves} or functions.
Concretely, we show how any continuous function on $[0,1]$ can be sampled
level by level, with each stage embedded in a \emph{meltdown-free} manner:
no re-labeling of older nodes, no partial expansions, and thus each partial
wave approximation remains \emph{immutable} as we move forward.

\subsection{Setup and Definitions}
\label{subsec:wave-setup}

\begin{definition}[Wave on an Interval]
  \label{def:wave}
  Throughout, a \emph{wave} is a continuous function
  \[
    f : [0,1] \;\to\; \mathbb{R}.
  \]
  We aim to approximate $f$ in a meltdown-free way: each ``level'' or ``depth''
  $d$ refines our approximation without ever altering previously assigned data.
\end{definition}

\paragraph{Uniform Sampling (Piecewise Constant).}
To keep matters concrete, we illustrate with a uniform sampling approach
that constructs piecewise constant approximations $W_d$ of $f$:
\begin{enumerate}
  \item \textbf{Partition $[0,1]$ at Depth $d$:}  
    Subdivide $[0,1]$ into $2^d$ subintervals, each of length $1/2^d$. Let
    \[
      S_d \;=\; \Bigl\{\bigl(x_{d,k},\,f(x_{d,k})\bigr)\;\Big|\;
        x_{d,k} = \tfrac{k}{2^d},\; k=0,\dots,2^d
      \Bigr\},
    \]
    capturing the function values at these sample points.
  \item \textbf{Approximation $W_d(x)$:}  
    Define $W_d(x)$ to be piecewise constant:
    \[
      W_d(x) \;=\; \sum_{k=0}^{2^d-1} \;
      \Bigl(\,
        f\bigl(x_{d,k}\bigr)
      \Bigr)\;
      \chi_{\left[\tfrac{k}{2^d},\,\tfrac{k+1}{2^d}\right)}(x),
    \]
    where $\chi_I(x)$ is the indicator function of interval $I$. By standard
    approximation theory, $W_d \to f$ uniformly as $d\to\infty$, thanks to $f$'s
    continuity on $[0,1]$.
\end{enumerate}

\begin{remark}[Other Schemes: Wavelet or Fourier Expansions]
  \label{rem:wavelets-fourier}
  One could replace uniform sampling with wavelet coefficients or partial
  Fourier sums. At depth $d$, let $S_d$ be the new block of coefficients
  or basis elements. Known convergence results still ensure $W_d \to f$ in
  $\ell^2$ or uniform norms, and the meltdown-free logic applies as we embed
  each block of coefficients at depth $d$ without disturbing old data.
\end{remark}

\subsection{Assigning Samples Meltdown-Free via Queueing}
\label{subsec:wave-embedding}

Recall from \S\ref{sec:prelim} that each depth $d\to d+1$ in the Geosodic Tree
adds $2^{d+1}$ fresh nodes (a pivot plus a perfect right subtree of depth $d$).
Denote these new nodes by
\[
  N_d \;=\;
  \{\,n_{d,1},\,n_{d,2},\,\dots,\,n_{d,\,2^{d+1}}\}.
\]
We define an \emph{injective} mapping $\phi_d$ that assigns each wave sample
to a unique new node, ensuring meltdown-free expansions.

\paragraph{Node Ordering and Queue of Samples.}
\begin{enumerate}
  \item \textbf{Order the new nodes $N_d$:}  
    In a meltdown-free tree, index $n_{d,1}\dots n_{d,2^{d+1}}$ in a fixed
    left-to-right or BFS manner. This yields a well-defined ordering of fresh nodes.
  \item \textbf{Maintain a global queue $\mathcal{Q}$ of unassigned samples:}  
    Initially empty. At depth $d$, do:
    \begin{itemize}
      \item \emph{Enqueue} the new wave samples $S_d$ onto $\mathcal{Q}$ (if any).
      \item \emph{Dequeue} up to $2^{d+1}$ items from $\mathcal{Q}$ (or fewer, if $\mathcal{Q}$ has fewer)
            in FIFO order. Assign them injectively to $n_{d,1}\dots n_{d,r_d}$, where
            $r_d = \min\bigl(2^{d+1}, \lvert\mathcal{Q}\rvert\bigr)$.  
      \item If $\lvert\mathcal{Q}\rvert > 0$ afterward, those leftover samples remain
            in $\mathcal{Q}$ for future depths.
    \end{itemize}
\end{enumerate}

\begin{definition}[Wave Sample Mapping $\phi_d$]
  \label{def:phi-d}
  Let $\phi_d\colon S_d \to N_d$ be the injective function that maps the
  \emph{dequeued} samples (from $S_d$ or prior leftover waves) to the newly
  created nodes $n_{d,i}\in N_d$ \emph{in the exact order} they were dequeued.
  By construction, each sample is assigned to a unique node, and no node is reused.
\end{definition}

\paragraph{Ensuring All Samples Eventually Assign.}
If $S_d$ is finite at each depth $d$ (true for uniform sampling or wavelet blocks),
then repeating this queue approach at $d=0,1,2,\dots$ eventually assigns every sample
to some finite depth. Each depth $d$ accommodates $2^{d+1}$ wave points, and
$\sum_{d=0}^{\infty} 2^{d+1} = \infty$ ensures no sample stays in $\mathcal{Q}$
indefinitely.

\subsection{Additive Construction of Wave Approximations}
\label{subsec:wave-additive}

Having assigned wave samples meltdown-free at each depth, we now define the piecewise
constant wave approximation $W_d$ \emph{immutablely} on $[0,1]$.

\begin{definition}[Immutable Wave Operator $\mathcal{A}_d$]
  Let $W_{d-1}$ be the wave approximation from depth $d{-}1$. Suppose $S_d$
  specifies new subintervals $[x_{d,k},\,x_{d,k+1})$ at depth $d$, each with
  midpoint-sample $f(x_{d,k})$. We define
  \[
    W_d \;=\; \mathcal{A}_d\bigl(W_{d-1},\,\phi_d(S_d)\bigr)\quad \text{on }[0,1]
  \]
  by explicitly combining the old approximation $W_{d-1}$ with the new values
  $f(x_{d,k})$. Concretely, for $x \in [0,1]$,
  \[
    W_d(x)
    \;=\;
    W_{d-1}(x)\,\Bigl(1 - \sum_{k=0}^{2^{d+1}-1}\chi_{\,[x_{d,k},\,x_{d,k+1})}(x)\Bigr)
    \;+\;
    \sum_{k=0}^{2^{d+1}-1} f\bigl(x_{d,k}\bigr)\,\chi_{\,[x_{d,k},\,x_{d,k+1})}(x),
  \]
  where $\chi_{I}(x)$ is the indicator function of interval $I$. In other words,
  $W_d(x)$ inherits $W_{d-1}(x)$ wherever the new partition does not refine,
  and it takes the new sample value $f(x_{d,k})$ on each newly introduced
  subinterval $[x_{d,k},\,x_{d,k+1})$. Since no old interval is overwritten,
  $W_{d-1}$ remains intact in all older subintervals, ensuring an ``additive''
  update and retaining meltdown-free immutability.
\end{definition}

\subsection{Convergence and the Link to Universal Enumeration}
\label{subsec:wave-conclusion}

\begin{theorem}[Uniform Convergence + Meltdown-Free Embedding]
\label{thm:uniform-wave-conv}
  Let $f$ be continuous on $[0,1]$. Using uniform sampling at depth $d$
  and the piecewise constant approximation $W_d(x)$, we have
  \[
    \|W_d - f\|_\infty \;\le\; \omega_f\bigl(\tfrac{1}{2^d}\bigr),
  \]
  where $\omega_f(\delta)$ is $f$'s modulus of continuity on $[0,1]$. Thus
  $W_d \to f$ uniformly as $d\to\infty$. Meanwhile, each wave sample is
  meltdown-free assigned via Definition~\ref{def:phi-d}, so no node at earlier
  depths is ever altered. Consequently, the wave approximations are
  \emph{additive} and \emph{immutable} at every stage.
\end{theorem}

\begin{proof}[Sketch]
  A standard result in approximation theory states that for a continuous
  function $f$ on $[0,1]$, sampling it on intervals of width $1/2^d$ yields
  piecewise constant approximations $W_d$ satisfying
  \[
    \|W_d - f\|_\infty \;\le\; \omega_f\bigl(\tfrac{1}{2^d}\bigr).
  \]
  Here, $\omega_f(\delta)$ is the modulus of continuity:
  \[
    \omega_f(\delta)
    \;=\;
    \sup\Bigl\{
      |f(x)-f(y)| : |x-y|\le\delta,\ x,y\in[0,1]
    \Bigr\}.
  \]
  Since $f$ is uniformly continuous on the compact interval $[0,1]$,
  $\omega_f(\delta)\to 0$ as $\delta\to 0$, implying $\|W_d - f\|_\infty\to0$
  and thus uniform convergence. 

  For meltdown-free embedding, we rely on the queue-based chunking:
  at depth $d$, the new samples $S_d$ are injectively mapped to $N_d$
  (Definition~\ref{def:phi-d}), leaving older nodes untouched. Hence
  each $W_{d-1}$ remains immutable within $W_d$, ensuring an additive,
  meltdown-free update at every stage.
\end{proof}

\paragraph{Relation to Universal Enumeration.}
This construction essentially \emph{enumerates} all wave samples
$\bigl(\cup_{d=0}^{\infty} S_d\bigr)$, each of which is finite or countable at
level $d$. By \emph{interleaving} or queueing them meltdown-free, we show these
countably many wave points embed exactly like any other discrete set in
Theorem~\ref{thm:univenum} (the universal enumeration property). Thus the
Geosodic Tree \emph{universally} accommodates continuous $f$ by sampling it
into a meltdown-free structure, bridging the discrete--continuous gap without
ever re-labeling old data.

\begin{remark}[All Continuous Functions at Once]
  By enumerating not just one function $f$, but partial approximations for
  \emph{every} $f$ in $\mathcal{C}([0,1])$, one could embed \emph{all} waves meltdown-free
  via a diagonal or fair-merge approach. Concretely, one might label each function $f_i$
  with depths $d=0,1,2,\dots$ and partial samples $S_d^{(i)}$, and then place those
  samples into a \emph{global} queue. Over infinitely many depths, a round-robin allocation
  draws from each $(i,d)$ in turn, assigning samples meltdown-free into newly created nodes.
  Although $\mathcal{C}([0,1])$ is uncountable, each individual $f_i$ has a \emph{countable}
  sequence of approximations, so the meltdown-free chunking can still accommodate them.
  We omit the details here, but it underscores the broad sweep of meltdown-free expansions
  beyond a single wave embedding.
\end{remark}

\smallskip

\noindent
\textbf{Conclusion:} By sampling any continuous wave $f$ at each depth, we obtain
immutable partial approximations $W_d$ that converge in the usual sense---and
all wave data is assigned meltdown-free to newly created nodes. This extends
the meltdown-free enumerations of discrete sets (\S\ref{sec:enumeration}) into a
\emph{rigorous wave-fitting framework}, reaffirming the Geosodic Tree as a
universal, layered structure for both discrete and continuous expansions.
