\subsection{The Principle of Geosodic Bounded Growth}
\label{subsec:bounded-growth}

We now highlight a geometric consequence of meltdown-free expansions:
\emph{doubling} occurs at each depth, even if we omit any arithmetic labeling.
While simple at first glance, this observation underpins the exponential
and logarithmic patterns that naturally emerge once we (optionally) interpret
the outer nodes in numeric coordinates.

\begin{corollary}[The Principle of Geosodic Bounded Growth]
\label{cor:geosodic-bounded-growth}
In a meltdown-free geosodic tree, each new depth $d+1$ attaches a fully formed
subtree of depth $d$ plus one pivot node (see Lemma~\ref{lem:no-smaller-step}).
Hence, the newly created nodes at depth $d+1$ \emph{double} those at depth $d$,
inducing a bounded-yet-exponential growth pattern in a purely \emph{geometric} sense:

\begin{itemize}
  \item No numeric labeling is assumed; the doubling follows from adjacency
  and minimal-step insertion alone.
  \item If, \emph{after the fact}, we assign in-order labels, the ``outermost''
  node at depth $d$ becomes label $2^{d-1}$, inverting to $d-1 = \log_2(\text{label})$.
\end{itemize}

Thus, without any arithmetic preconditions, an exponential--log relationship
emerges \emph{intrinsically} from meltdown-free expansions.
\end{corollary}

\begin{proof}[Sketch]
By the meltdown-free principle (Definition~\ref{def:geosodic-tree} and
Lemma~\ref{lem:no-smaller-step}), moving from depth $d$ to $d+1$ must
add exactly $2^{d+1}$ fresh nodes (a pivot plus a perfect subtree of depth~$d$).
Hence the number of new nodes \emph{doubles} each level, in a purely spatial
(adjacent) sense, with no relabeling or partial insertions.
If we later choose to label nodes in-order, it is straightforward
to see that the outer node at depth $d$ takes label $2^{d-1}$ (e.g.\ 
root at depth~1 is label~1, next level is label~2, etc.), yielding an
exponential curve in one axis and a log when inverted.
\end{proof}

\paragraph{Remark.}
We stress that this bounded growth is \emph{not} a numerical artifact:
it arises from the \emph{structure} of pivot+subtree expansions.
No arithmetic or labeling is needed \emph{a priori}. The doubling
pattern, and thus its exponential/log inversion, follows inherently
from meltdown-free adjacency constraints.
