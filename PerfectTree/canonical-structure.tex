\subsection{Uniqueness and Canonicity of the Geosodic Tree}
\label{subsec:uniqueness}

We now show that the \emph{Geosodic Tree} is essentially \emph{the} unique
(meltdown-free) structure that remains perfectly balanced at each depth,
expanding from depth $d$ to $d+1$ in a single step with no partial
insertions or re-labeling. This establishes its \emph{canonicity} among such
trees.

\begin{theorem}[Uniqueness of the Geosodic Tree]
\label{thm:geosodic-uniqueness}
Any rooted binary tree that
\begin{enumerate}
  \item maintains a \emph{perfect} shape at each depth,
  \item expands from depth $d$ to $d+1$ in one step, adding no partial insertions,
  \item is \emph{meltdown-free} (no re-labeling of existing nodes),
\end{enumerate}
is isomorphic to the Geosodic Tree. Moreover, its expansion from depth $d$ to
$d+1$ is uniquely determined as adding a new root (pivot) plus a perfect right
subtree of depth $d$.
\end{theorem}

\begin{proof}[Proof (Strong Induction on Depth)]
We prove by strong induction on the depth $d$ that under these conditions,
the resulting tree of depth $d$ must be isomorphic to the Geosodic Tree of
depth $d$.

\paragraph{Base Case ($d=0$).}
A perfect binary tree of depth $0$ has exactly $2^{0+1}-1 = 1$ node (just
the root). The Geosodic Tree at depth $0$ likewise has one root node and no
other nodes. Therefore, any meltdown-free, perfectly balanced tree of depth~0
is trivially isomorphic to the Geosodic Tree of depth~0.

\paragraph{Inductive Hypothesis.}
Assume that for some depth $d \ge 0$, \emph{any} meltdown-free, perfectly
balanced tree of depth $d$ that expands in a single step (no partial
insertions) is isomorphic to the Geosodic Tree of depth $d$.

\paragraph{Inductive Step ($d \to d+1$).}
Let $T$ be a meltdown-free, perfectly balanced tree of depth $d+1$. We must show
$T$ is isomorphic to the Geosodic Tree of depth $d+1$.

\begin{enumerate}
  \item \textbf{Node Counts.}
    A perfect binary tree of depth $d$ contains $2^{d+1}-1$ nodes. A perfect tree
    of depth $d+1$ must have $2^{(d+1)+1}-1 = 2^{d+2} - 1$ nodes. Hence, going
    from depth $d$ to $d+1$ requires adding
    \[
      \bigl(2^{d+2}-1\bigr)\;-\;\bigl(2^{d+1}-1\bigr)
      \;=\;
      2^{d+1}
    \]
    new nodes in one step to maintain perfect balance.

  \item \textbf{Structure at Depth $d+1$.}
    Since $T$ is meltdown-free and expands in a single step from depth $d$ to
    $d+1$, it must introduce exactly $2^{d+1}$ \emph{new} nodes simultaneously
    (no partial insertions). Let $T_d$ be the subtree of $T$ up to depth $d$
    (the “old part”). By the inductive hypothesis, $T_d$ is isomorphic to the
    Geosodic Tree of depth $d$.

    Because $T$ is perfectly balanced at depth $d+1$, it has one new root node
    (pivot) \emph{that is distinct from any node in $T_d$}, with $T_d$ as its
    left subtree. To complete the perfect shape, the newly attached right subtree
    must also contain $2^{d+1}-1$ nodes (so that the total is $2^{d+2}-1$ at
    depth $d+1$). That right subtree is itself a perfect binary tree of depth $d$.

  \item \textbf{Isomorphism to the Geosodic Tree.}
    This construction---a new pivot node on top, the old depth-$d$ tree on the
    left, and a perfect subtree of depth $d$ on the right---is precisely how the
    Geosodic Tree of depth $d+1$ is formed. Namely, the Geosodic Tree also
    attaches a new pivot plus a perfect right subtree of depth $d$ to
    its (already isomorphic) depth-$d$ tree. Thus $T$ matches the Geosodic
    structure at depth $d+1$ up to node renaming, i.e.\ they are isomorphic.
\end{enumerate}

By strong induction, any meltdown-free, perfectly balanced tree of depth $d$,
expanded in single steps with no partial insertions, is isomorphic to the
Geosodic Tree of the same depth.
\end{proof}

\paragraph{Canonical Implications.}
Theorem~\ref{thm:geosodic-uniqueness} shows there is essentially \emph{no other}
meltdown-free tree that remains perfectly balanced at each depth and grows
exactly one step from $d$ to $d+1$. Thus the Geosodic Tree can be viewed as the
\emph{canonical} (or unique) structure under these constraints. In particular,
any different choice of adding nodes or rearranging them at intermediate depths
would break meltdown-free conditions, fail perfect balance, or require partial
insertions---none of which is allowed. Consequently, the minimal-step growth
(pivot + perfect subtree) is forced. 

\emph{This uniqueness has important implications for representing both discrete
enumerations and continuous functions. Since the structure of the tree is
uniquely determined by the constraints, the embedding of any given enumeration
or sampling scheme is also uniquely determined (up to the initial ordering of
nodes at each level). This establishes the Geosodic Tree as a truly canonical
framework for meltdown-free expansions.}
