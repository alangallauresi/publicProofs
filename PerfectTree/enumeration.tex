\section{Universal Enumeration Theorem}
\label{sec:enumeration}

We now present our main result: a \emph{Geosodic Tree} (see \S\ref{sec:prelim})
can embed \emph{any} countably infinite discrete family in a minimal-growth, meltdown-free manner.

\subsection{Main Theorem and Proof}

\begin{theorem}[Universal Enumeration]
\label{thm:univenum}
Let $\{C_k\}_{k=0}^\infty$ be any infinite discrete enumeration (e.g.\ G\"odel codes, Gray codes, 
or natural numbers). Then there exists a geosodic tree in which each $C_k$ is assigned to a unique node 
at some finite depth, with no re-labeling or collisions.
\end{theorem}

\begin{proof}[Proof via ``Chunking'']
We construct the geosodic tree \emph{level by level}, assigning chunks of unassigned codes 
to the newly created nodes at each stage.

\paragraph{Stage 0 (Depth 0).}
Initialize the tree at depth $0$ with a single root node (total $2^{0+1}-1 = 1$ node). 
Assign $C_0$ (the first code) to this root. Thus, after stage 0, 
exactly one code $C_0$ has been placed.

\paragraph{General Stage (Depth $d \to d+1$).}
Assume we have built a geosodic tree of depth $d$, with $2^{d+1}-1$ total nodes, 
and assigned $\{C_0, \dots, C_m\}$ to distinct nodes. 
To form depth $d+1$, we add:
\begin{itemize}
  \item One \emph{pivot node} on top (the new root), and
  \item A perfect right subtree of depth $d$ containing $2^{d+1}-1$ new nodes.
\end{itemize}
Hence we introduce a total of $2^{d+1}$ \emph{new} nodes in a single meltdown-free step, 
bringing the total node count to $(2^{d+1}-1) + 2^{d+1} = 2^{d+2}-1$ at depth $d+1$.

\smallskip
\emph{Chunking the codes:}
- Let $r_d = 2^{d+1}$ be the number of newly created nodes at stage $d+1$.
- Take the next $r_d$ \emph{unassigned} codes, namely $\{C_{m+1}, \dots, C_{m + r_d}\}$.
- Assign each of these $r_d$ codes \emph{exactly once} to the newly created nodes at depth $d+1$.

If fewer than $r_d$ codes remain unassigned, we simply place all of them now; 
any new node that does not get a code remains “empty” (unused) at that stage, which is allowed 
because no old labels are re-labeled and no partial insertion occurs. 
If more than $r_d$ codes remain, we place exactly $r_d$ of them and move on.
In either case, $m$ increases by however many codes are placed at this level.

\paragraph{Remark on Irregular Sequences.}
This chunking strategy does \emph{not} require $\{C_k\}$ to be arranged 
in contiguous or numeric order. Even if indices jump widely (e.g.\ $C_0, C_{1000}, C_2,\dots$),
we place whichever unassigned codes arise into the $r_d = 2^{d+1}$ fresh nodes.
Thus, any “sparse” or out-of-order enumeration still gets fully embedded 
without altering previously assigned labels.

\paragraph{Why No Collisions or Re-labeling?}
- \textbf{Collisions:} Each newly created node receives \emph{at most one} code, 
  distinct from codes placed at earlier depths. Thus, collisions never occur.
- \textbf{No Re-labeling (Meltdown-Free):} We add \emph{only new} nodes. 
  Older nodes and labels remain intact (no partial insertions), so the expansion is meltdown-free.

\paragraph{Completeness of Assignment (Surjectivity).}
We claim that \emph{every} code $C_k$ is eventually assigned to some node at a finite depth.
Indeed, at depth $d$, we place exactly $r_d = 2^{d+1}$ codes (unless fewer remain). Thus,
\[
  \sum_{d=0}^{\infty} 2^{d+1} 
  \;=\; 
  2 \sum_{d=0}^{\infty} 2^d 
  \;=\; 
  \infty,
\]
so there is \emph{sufficient capacity} to accommodate infinitely many codes. 
Concretely, for any $k$, pick $D$ large enough so that 
$\sum_{d=0}^D 2^{d+1} \;\ge\; k$, ensuring $C_k$ appears in one of the next chunks. 
Hence every code $C_k$ is assigned exactly once at some finite stage $d$, 
and \emph{every} object $\{C_k\}$ appears in the tree.

\smallskip
\noindent
Therefore, this procedure preserves perfect balance (the pivot + subtree step) 
and avoids re-labeling at each level, embedding the entire countably infinite family 
in a meltdown-free manner, as claimed.
\end{proof}

\noindent
\textbf{Conclusion of Theorem~\ref{thm:univenum}:}
Because each depth-$d$ step adds $2^{d+1}$ new nodes and assigns up to $2^{d+1}$ codes,
the geosodic tree \emph{universally enumerates} any countably infinite set. 
No old labels are disturbed, no collisions occur, and the minimal meltdown-free increments 
uphold perfect balance throughout.

