\section{A Surprising \texorpdfstring{$-1/12$}{-1/12} Ratio Identity}
\label{sec:ratio}

Even in a purely finite, combinatorial setting, the \emph{geosodic tree} 
reveals a striking numeric identity when labeled in-order:

\begin{theorem}[$-1/12$ Ratio Identity]
\label{thm:minus-twelfth}
Consider a geosodic tree of depth $d$, where the nodes are labeled in-order 
from $1$ to $2^{d+1}-1$. Let $S_{\text{left}}$ be the sum of labels in the 
left subtree, $S_{\text{right}}$ the sum in the right subtree, and 
$S_{\text{excl}}$ the sum of all labels excluding the root. Then
\[
\Bigl(\frac{S_{\text{left}}}{S_{\text{excl}}}\Bigr)
\;-\;
\Bigl(\frac{S_{\text{left}}}{S_{\text{right}}}\Bigr)
\;=\;
-\tfrac{1}{12}.
\]
\end{theorem}

\begin{proof}[Sketch]
(Algebraic steps as in your existing proof, showing 
\(\tfrac{S_{\text{left}}}{S_{\text{excl}}} = \frac{1}{4}\),
\(\tfrac{S_{\text{left}}}{S_{\text{right}}} = \frac{1}{3}\).)
\end{proof}

\paragraph{Connection to ``\texorpdfstring{$-1/12$}{-1/12}'' in Analysis.}
The appearance of $-\tfrac{1}{12}$ here \emph{echoes} the well-known 
(infinite-sum) statement $\,1 + 2 + 3 + \dots = -\tfrac{1}{12}$ 
as understood through zeta function regularization or Ramanujan summation. 
\emph{However}, in our setting, the constant emerges in a \emph{purely finite} 
tree-based summation, with no direct invocation of analytic continuation. 
This suggests an intriguing \emph{resonance} with zeta-related methods 
in number theory, but we do not claim a full equivalence or deeper analytic link. 
Exploring whether similar balanced expansions might provide further insights 
into zeta function properties is an open direction. For now, we simply highlight 
the combinatorial coincidence that $-\tfrac{1}{12}$ surfaces in this 
finite-tree ratio---underscoring how classical constants can arise 
in unexpected discrete contexts.
