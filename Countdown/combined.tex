\documentclass[11pt]{article}

\usepackage[utf8]{inputenc}
\usepackage{amsmath, amsthm, amssymb}
\usepackage{fullpage}
\usepackage{hyperref}
\usepackage{graphicx}

\hypersetup{
  colorlinks=true,
  linkcolor=blue,
  citecolor=blue,
  urlcolor=blue
}

\newtheorem{theorem}{Theorem}
\newtheorem{lemma}[theorem]{Lemma}
\newtheorem{corollary}[theorem]{Corollary}
\newtheorem{proposition}[theorem]{Proposition}

\theoremstyle{definition}
\newtheorem{definition}[theorem]{Definition}

\theoremstyle{remark}
\newtheorem*{remark}{Remark}

\title{Countdown, Geosodic Expansion, \\
and an Application to Ethical AI}
\author{Alan Gallauresi}
\date{\today}

\begin{document}
\maketitle

\section{Introduction}
\label{sec:intro}

The classic \emph{Countdown numbers game} can be viewed as a bounded
constraint satisfaction problem (CSP), wherein one attempts to reach a target
number $T$ using at most six given numbers with the operations
$\{ +, -, \times, \div \}$.

In this short paper, we show how the \emph{finite search space} of Countdown
naturally embeds into a ``meltdown-free'' or ``geosodic'' expansion framework,
initially introduced in our companion paper~\cite{geosodicArxiv}, wherein no
prior constraints are re-labeled or destroyed (``no meltdown''), and each new
``depth'' of the tree expands in a carefully structured, pivot-based manner.

Beyond Countdown, we also illustrate how this \emph{geosodic} viewpoint applies
to AI ethics decisions: the meltdown-free layering ensures old moral constraints
cannot be discarded, yet the system can keep refining its solutions within a
bounded or tolerance-driven approach.
 \section{Preliminaries}
\label{sec:prelims}

\subsection{Countdown as a Finite CSP}
We briefly recall the \emph{Countdown numbers game}:
\begin{itemize}
  \item We have six initial numbers, e.g.\ $n_1, \dots, n_6$.
  \item We have a target integer $T$.
  \item We may use each number at most once.
  \item Allowed operations: $\{+, -, \times, \div\}$ (no division by zero).
\end{itemize}
We can form an expression by combining any two available numbers (or partial
expressions) at a time, until at most five operations have been performed.
Because there is a finite set of combinations, this is a bounded CSP.

\subsection{Geosodic (Meltdown-Free) Expansion}
We rely on the meltdown-free tree construction detailed in~\cite{geosodicArxiv}.
In essence:
\begin{itemize}
  \item At depth $d=0$, there is one root node.
  \item Moving from depth $d$ to $d+1$ adds exactly $2^{d+1}$ nodes:
        a new pivot (root) plus a perfect right subtree of depth $d$.
  \item No node from a previous level is re-labeled or shuffled (no ``meltdown'').
\end{itemize}
This yields a perfectly balanced tree at each depth, providing a unique
structured form for incremental expansions.
 \section{Countdown in a Geosodic Subtree: Proof of Concept}
\label{sec:proof}

\begin{theorem}[Countdown Embeds in a Bounded Geosodic Subtree]
\label{thm:countdown-embed}
Let $n \le 6$ be the number of initial values and $T$ a target.
Then all expressions formed by combining these $n$ values with at most
$(n-1)$ operations can be represented in a meltdown-free subtree of
depth $d = (n-1)$ of a geosodic expansion.
\end{theorem}

\begin{proof}[Sketch of Proof]
For $n=6$, we have at most five operations. We reserve a geosodic subtree
of depth $5$. Each node at depth $k \le 5$ represents a partial expression
that has used exactly $k$ operations. The unique meltdown-free expansions
guarantee:
\begin{itemize}
  \item No re-labeling of previous partial expressions.
  \item Perfectly balanced structure to hold all possible combinations.
\end{itemize}
Hence, each valid Countdown expression with $k$ steps is assigned to a
distinct node in the depth-$k$ layer. By depth $5$, we have enumerated all
possible outcomes. Full details appear in~\cite{geosodicArxiv}.
\end{proof}

\subsection{Time Complexity and Tolerance Argument}
Even though raw CSP search can be exponential, the meltdown-free approach
organizes the partial expressions in a layered fashion, preventing excessive
redundancy. For well-structured CSPs (like Countdown) or AI decisions with
tolerance bounds, this yields a more tractable search in practice.

\begin{definition}[Ethical CSP and Tolerance]
An \emph{ethical CSP} is one where partial solutions must obey certain
fixed moral constraints (no meltdown of previous ethical commitments),
and we allow near-solutions if the ``cost'' is within some tolerance $\delta$.
\end{definition}

This parallels the approximate acceptance of near-target expressions in
Countdown. As the geosodic structure grows, deeper expansions systematically
refine solutions without discarding earlier valid constraints.
 \section{Conclusion and Future Work}
\label{sec:conclusion}

We have shown that the finite search space of Countdown naturally maps into
a meltdown-free (geosodic) expansion, ensuring no contradictory re-labeling
of partial expressions. This same approach extends to \emph{ethical CSPs} in
AI, where tolerances guide incremental moral refinements without discarding
established constraints.

\paragraph{Future Work.} We foresee further applications of meltdown-free
structures to other bounded decision problems, and deeper explorations of
moral consistency in AI. In all such settings, the pivot-plus-perfect-subtree
pattern provides a canonical way to grow solutions in stable, no-meltdown steps.
 
\bibliographystyle{plain}
\bibliography{references}

\end{document}
